%% First attempt at 'manually' drawing a Universal Plotting Sheet,
%% using LaTeX and tikz
%% Author: Kevin Zembower

\documentclass[tikz,border=1cm]{standalone}
\usepackage{tikz}
\tikzset{font={\fontsize{8pt}{12}\selectfont}}

\begin{document}

\begin{centering}

% Define a few constants for easy configuration
  \def\radius{6.5cm}            %Scale everything off of the radius
  \def\onedegrad{\radius*0.975} %Distance in from radius circle to end of one-degree line
  \def\fivedegrad{\radius*0.95} %Distance in from radius circle to end of 5-degree line
  \def\tendegrad{\radius*0.925} %Distance in from radius circle to end of 10-degree line
  \def\labelrad{\radius*0.90}   %Distance in from radius circle to 10-degree label text
  \def\outerlabelrad{\radius*1.0375} %Distance to outer degree label text

  \begin{tikzpicture}[scale=1,
      degree lines/.style={color=gray, very thin}]
    
    \draw (0,0) circle (\radius); %Outer circle
    
    % lines at every degree
    \foreach \x in {0,...,359} \draw [degree lines] (\x:\onedegrad) -- (\x:\radius);

    % lines at every 5 degrees
    \foreach \x in {0,5,...,355}  \draw [degree lines] (\x:\fivedegrad) -- (\x:\radius);
    
    % labels and longer lines at every 10 degrees
    \foreach \x in {0,10,...,350} {
      \node[scale=1, rotate=-\x] at (360-\x+90:\labelrad) {\x};
      \draw (\x:\tendegrad) -- (\x:\radius);
    };

    %% Draw the outer text labels, from 10-70 degrees, CW and CCX from 90 degrees:
    \foreach \x in {10,20,...,70} {
      \node[scale=0.75, rotate=\x] at (\x:\outerlabelrad) {\x};
      \node[scale=0.75, rotate=-\x] at (-\x:\outerlabelrad) {\x};
    }
    
    %% Draw the latitude lines
    \foreach \y in {-\radius*2,-\radius*1,0,\radius*1,\radius*2} {
      \draw (-\radius*1.5,\y) -- (\radius*1.5,\y);
    }
    
    %% Draw the center vertical line, and scale
    \draw (0, -\radius*2) -- (0, \radius*2);
    \foreach \x in {-2,-1,...,1} {
      %% Draw the one degree lines
      \foreach \tick in {1,2,...,59} { %Don't draw on top of existing lines
        \draw [degree lines] (0, {\radius*\x+(\tick*\radius/60)}) --
        (\radius-\onedegrad, {\radius*\x+(\tick*\radius/60)});
      }
      %% Draw the 5 degree lines
      \foreach \tick in {5,10,...,55} {
        \draw [degree lines] (0, {\radius*\x+(\tick*\radius/60)}) --
        (\radius-\fivedegrad, {\radius*\x+(\tick*\radius/60)});
      }
      %% Draw the 10 degree lines and numbers
      \foreach \tick in {10,20,...,50} {
        \draw [degree lines] (0, {\radius*\x+(\tick*\radius/60)}) --
        (\radius-\tendegrad, {\radius*\x+(\tick*\radius/60)});
        \node [scale=1] at (\radius-\labelrad,
              {\radius*\x+(\tick*\radius/60)}) {\tick}; 
      }
    }
           
    %% Draw the latitude scale in the bottom right-hand corner
    \def\zeroy{\radius*-2}; %Set a new reference point, for 0,0.
    \def\zerox{\radius*1.25};
    \def\scalex{\radius/0.60}; % Width of lattitude scale. MUST BE
                               % equal to \radius. Divide by 0.60,
                               % because latitude scale is only 60
                               % degree wide.
    \def\scaley{\scalex*0.75}; %Height of latitude scale, arbitrary

    \foreach \step in {0.00,0.02,0.04,0.06,0.08,0.1,0.2,0.3,0.4,0.5,0.6} { %for each of the curved lines
      \foreach \y in {0.0,0.01,...,0.69} {   %for each degree in the vertical scale
        \draw ({((-sin(90-\y*100))*\step*\scalex)+\zerox},{(\y*\scaley)+\zeroy}) --
        ({((-sin(90-(\y+0.01)*100))*\step*\scalex)+\zerox},{((\y+0.01)*\scaley)+\zeroy});
      }
    }
    \foreach \lat in {0, 10,...,70} {  %For each of the horizontal latitude lines
      \draw ({(-sin(90-\lat)*.6*\scalex)+\zerox},{(\lat/100*\scaley)+\zeroy}) --
      (\zerox, {(\lat/100*\scaley)+\zeroy}) node [right] {\lat};
    }
    
    \foreach \lat in {5,15,...,65} {  %For each of the thinner horizontal latitude lines
      \draw[very thin, gray] ({(-sin(90-\lat)*.6*\scalex)+\zerox}, {(\lat/100*\scaley)+\zeroy}) --
      (\zerox,{(\lat/100*\scaley)+\zeroy});
    }
    \foreach \min in {0,10,...,50} {     %For each minute in the latitude scale
      \node [above left] at ({((-\min/10)*(\scalex/10))-(\scalex/10)+\zerox}, \zeroy) {\min};
    }

    %% Enter the label for where to find this file
    \node [anchor=south west] at (-\radius*1.5,-\radius*2)
          {\LaTeX{} and TikZ Universal Plotting Sheet v0.1};
    
\end{tikzpicture}
\end{centering}

\end{document}
