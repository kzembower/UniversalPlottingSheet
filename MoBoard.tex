%% First attempt at 'manually' drawing a Manouvering Board, using LaTeX and tikz
%% Author: Kevin Zembower

% A simple compass
% Author: Dario Orescanin
%% From website: http://www.texample.net/tikz/examples/degree-wheel/ downloaded on 2 Apr 2020.

\documentclass{minimal}
\usepackage{tikz}
\begin{document}

\begin{centering}

% Define a few constants for easy configuration
\def\radius{2cm}
\def\onedegrad{1.8cm}
\def\fivedegrad{1.75cm}
\def\tendegrad{1.7cm}
\def\labelrad{1.6cm}

\begin{tikzpicture}[scale=4]

  % adding a subtle gray tone to add a bit of "personality"
  \shade[shading=radial, inner color=white, outer color=gray!15] (0,0) circle (\radius);

  \draw (0,0) circle (\radius);
  \draw[fill=black] (0,0) circle (.02mm);
  \node[draw, circle, inner sep=.2mm] (a) at (0,0) {};

  % helper lines
  \foreach \x in {0, 45, ..., 360} \draw[very thin, gray!40] (a) -- (\x:\radius);

  % main lines
  \foreach \x in {0,...,359} \draw (\x:\onedegrad) -- (\x:\radius);

  % labels and longer lines at every 10 degrees
  \foreach \x in {0,10,...,350}
  {
    \node[scale=1.4, rotate=\x*-1] at (360-\x+90:\labelrad) {\x};
    \draw (\x:\tendegrad) -- (\x:\radius);
  };

  % lines at every 5 degrees
  \foreach \x in {0,5,...,355}  \draw (\x:\fivedegrad) -- (\x:\radius);

\end{tikzpicture}
\end{centering}

\end{document}
