%% First attempt at 'manually' drawing a Manouvering Board, using LaTeX and tikz
%% Author: Kevin Zembower

% A simple compass
% Author: Dario Orescanin
%% From website: http://www.texample.net/tikz/examples/degree-wheel/ downloaded on 2 Apr 2020.
\documentclass[tikz,border=10pt]{standalone}

%\documentclass{minimal}
\usepackage{tikz}
%\usetikzlibrary{calc}

\begin{document}

\begin{centering}

% Define a few constants for easy configuration
  \def\radius{8.5cm}            %Scale everything off of the radius
  \def\onedegrad{\radius*0.975}
  \def\fivedegrad{\radius*0.95}
  \def\tendegrad{\radius*0.925}
  \def\labelrad{\radius*0.90}

\begin{tikzpicture}[scale=1,
    degree lines/.style={color=gray, very thin}]
  
  \draw (0,0) circle (\radius); %Outer circle

  % lines at every degree
  \foreach \x in {0,...,359} \draw [degree lines] (\x:\onedegrad) -- (\x:\radius);

  % lines at every 5 degrees
  \foreach \x in {0,5,...,355}  \draw [degree lines] (\x:\fivedegrad) -- (\x:\radius);

  % labels and longer lines at every 10 degrees
  \foreach \x in {0,10,...,350}
  {
    \node[scale=1, rotate=\x*-1] at (360-\x+90:\labelrad) {\x};
    \draw (\x:\tendegrad) -- (\x:\radius);
  };

  %% Draw the latitude lines
  \foreach \y in {-\radius*2,-\radius*1,0,\radius*1,\radius*2}
  \draw (-\radius*1.5,\y) -- (\radius*1.5,\y);

  %% Draw the center vertical line, and scale
  \draw (0, -\radius*2) -- (0, \radius*2);
  \foreach \x in {-2,-1,...,1}
           {
             %% Draw the one degree lines
             \foreach \tick in {1,2,...,59} %Don't draw on top of existing lines
                      {
                        \draw [degree lines] (0, {\radius*\x+(\tick*\radius/60)}) --
                        (\radius-\onedegrad, {\radius*\x+(\tick*\radius/60)});
                      }
             %% Draw the 5 degree lines
             \foreach \tick in {5,10,...,55}
                      {
                        \draw [degree lines] (0, {\radius*\x+(\tick*\radius/60)}) --
                        (\radius-\fivedegrad, {\radius*\x+(\tick*\radius/60)});
                      }
             %% Draw the 10 degree lines and numbers
             \foreach \tick in {10,20,...,50}
                      {
                        \draw [degree lines] (0, {\radius*\x+(\tick*\radius/60)}) --
                        (\radius-\tendegrad, {\radius*\x+(\tick*\radius/60)});
                        \node [scale=1] at (\radius-\labelrad,
                              {\radius*\x+(\tick*\radius/60)}) {\tick}; 
                      }
           }
           
           %% Draw the latitude scale in the bottom right-hand corner
           \def\latzeroy{\radius*-2}; %Set a new reference point, for 0,0.
           \def\latzerox{\radius*1.25};
           \def\lath{\radius*0.75}; %Height of latitude scale
           \def\latw{\radius*0.75}; %Width of lattitude scale
           \draw (\latzerox, \latzeroy) -- (\latzerox, \latzeroy+\lath);
           %% Draw the horizontal lines
           \foreach \y in {0,10,...,70} {
             \draw (\latzerox, {\latzeroy+\lath*\y/70}) --
             (\latzerox-5, {\latzeroy+\lath*\y/70});
           }
           %% \draw (\latzerox-\latw, \latzeroy) cos(\latzerox
\end{tikzpicture}
\end{centering}

\end{document}
