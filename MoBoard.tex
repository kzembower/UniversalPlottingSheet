%% First attempt at 'manually' drawing a Manouvering Board, using LaTeX and tikz
%% Author: Kevin Zembower

% A simple compass
% Author: Dario Orescanin
%% From website: http://www.texample.net/tikz/examples/degree-wheel/ downloaded on 2 Apr 2020.
\documentclass[tikz,border=10pt]{standalone}

%\documentclass{minimal}
\usepackage{tikz}
%\usetikzlibrary{calc}

\begin{document}

\begin{centering}

% Define a few constants for easy configuration
  \def\radius{8.5cm}            %Scale everything off of the radius
  \def\onedegrad{\radius*0.975}
  \def\fivedegrad{\radius*0.95}
  \def\tendegrad{\radius*0.925}
  \def\labelrad{\radius*0.90}

\begin{tikzpicture}[scale=1,
    degree lines/.style={color=gray, very thin}]
  
  \draw (0,0) circle (\radius); %Outer circle

  % main lines
  \foreach \x in {0,...,359} \draw [degree lines] (\x:\onedegrad) -- (\x:\radius);

  % labels and longer lines at every 10 degrees
  \foreach \x in {0,10,...,350}
  {
    \node[scale=1, rotate=\x*-1] at (360-\x+90:\labelrad) {\x};
    \draw (\x:\tendegrad) -- (\x:\radius);
  };

  % lines at every 5 degrees
  \foreach \x in {0,5,...,355}  \draw [degree lines] (\x:\fivedegrad) -- (\x:\radius);

  %% Draw the latitude lines
  \foreach \y in {-\radius*2,-\radius*1,0,\radius*1,\radius*2}
  \draw (-\radius*1.5,\y) -- (\radius*1.5,\y);
  
\end{tikzpicture}
\end{centering}

\end{document}
